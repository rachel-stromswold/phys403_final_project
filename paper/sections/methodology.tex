%\section{Data} \label{sec:develop}

Historic data is taken from merger catalogs from the second and third year of LIGO observations \cite{GWTC_2,GWTC_3}. Our analysis is only interested in the solid angle covered by sky localization, $\Omega$ and distances obtained from the GW data $d_L$ as well as their uncertainties, $\sigma_d$. These uncertainties are taken to be Gaussian. We marginalize over all other parameters to obtain a three dimensional probability distribution containing only these parameters of interest.

We generate simulated events by estimating this probability distribution and performing MCMC sampling implemented in the emcee package. Details of the distribution estimation and sampling are given in appendix \ref{append_a}.

After producing simulated GW events we need to select a suitable sky catalog. These data are also simulated by randomly sampling a number of galaxy clusters which are assumed not to interact with one another. The total mass of the cluster is assumed to follow a Press-Schechter mass distribution\cite{Press_1974}. Peculiar motions (which affects the measured redshift $z$) and distance from the center of mass are sampled such that the Virial theorem is obeyed, but other important factors like the very high concentrations of dark matter within clusters are neglected. More details on cluster generation are given in appendix \ref{append_b}.
    
%\subsection{\label{Method} Methodology}

We start with a simple application of Bayes' theorem,
\begin{align}
    p(H_0|d_{GW}, d_C)&\propto p(d_{GW}, d_C|H_0)p(H_0)\nonumber\\
    &= p(d_{GW}|H_0)p(d_C|H_0)p(H_0)
\label{eq:bayes}
\end{align}.
where $d_{GW}$ refers to the observed data from the gravitational wave detectors and $d_C$ refers to data from the known catalog of stars and their redshifts. We break this joint likelihood into a product of likelihoods since both experiments are statistically independent.

Since we are considering dark sirens without electromagnetic counterparts, we do not know the host galaxy in which the merger occurred. Thus we must marginalize over all possible galaxies within the volume of possible locations supplied by the GW data. Since the number of galaxies is discrete, this is expressed as a sum.

\begin{align}
    p(&d_{GW}, d_C|{z_j, \hat{\Omega}_j},H_0)  \propto  \nonumber\\
    &\sum_{i} w_i \int dV p(d_{GW}|d_L, \hat{\Omega}_{GW}) 
    p(d_C|{z_, \hat{\Omega}_j}) \nonumber\\
    & \delta_D (d_L - d_L(z_i, H_0)) \delta_D (\hat{\Omega}_{GW} - \hat{\Omega}_i)
    \label{eq:marginal_like}
\end{align}
In principle, we should apply a weight to each galaxy $w_i$ we include this in the expression below, but we take for simplicity we take $w_i=1$ for all galaxies within the volume of interest and $w_i=0$ for all galaxies outside. This same approximation has been used in previous analyses\cite{Chen_2018,GW170814_DES,Nair_2018}. To evaluate this integral we switch to spherical coordinates and work in redshift space as opposed to distance space. This results in a change of coordinates

\begin{equation}
dV \propto \frac{d^2 V}{d z_i d\hat{\Omega}_i} \frac{d z_i}{d r} d r d\hat{\Omega} \propto \frac{r^2(z_i)}{H(z_i)}
\end{equation}

where $H(z_i)$ is the Hubble parameter in redshift space,

\begin{equation}
H(z_i) = H_0^3 \left(\Omega_m (1+z)^3 + \Omega_\Lambda\right)
\end{equation}
where we have assumed a flat universe so $\omega_k = 0$. We will take $\omega_m = 0.3$ and $\omega_\Lambda = 0.7$ in keeping with Soares-Santos et al.We expect our catalog of galaxies to have very low uncertainty on sky location, so we take $p(d_C|z_i,\Omega)=p(d_C|z_i)\delta(\Omega - \Omega_i)$. Finally we can write the posterior distribution we evaluate.

\begin{align}
     p&(H_0|d_{GW}, d_C)\propto p(H_0) \sum_i \frac{1}{Z_i} \nonumber\\
     &\times \int dz_i p\left(d_{GW}|d_L(z_i, H_0), \hat{\Omega_{i}}\right)p(d_C|z_i)\frac{r^2 (z_i)}{H(z_i)}
    \label{eq: finaleq}
\end{align}
where $d_L(z_i, H_0) = cz_i / H_0$ uses Hubble's law to express luminosity distance in terms of redshift and an assumed value for $H_0$. We assume that the errors redshifts are Gaussian, in keeping with\cite{Singer_2016}. Under this assumption,

\begin{align}
    p&\left(d_{GW}|d_L(z_i, H_0)\right)\propto \nonumber\\
    &\frac{p(\hat{\Omega})}{\sqrt{2\pi}\sigma(\hat{\Omega})} \exp\left(-\frac{(cz_i/H_0 - \hat{d}_{GW})^2}{2\sigma_{GW}^2}\right)
    \label{eq:GW_like}
\end{align}

Since the generated data does note need the error correction of the GW missing part.  The final equation of \ref{eq:posterior}which included the evidence term should become:

