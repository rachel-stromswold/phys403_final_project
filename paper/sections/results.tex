\section{Results} \label{sec:conclusions}
We produce simulated events until we find at least $100$ with sufficiently small localizations that further analysis is practical.
This cutoff is arbitrary. For all results reported, we used a maximum volume of $1\times 10^{9}$ \si{Mpc^3}.
We also observed values which had anomalously small localization volumes so we constrained all simulated events to have a localization volume larger than $1\times 10^{6}$ \si{Mpc^3}.
We found that, on average, a total of $484$ observations were needed to produce the desired $100$ well localized events.
These localized regions corresponded to $20.66\%$ of all generated samples.
In all of our tests we injected a known value of $H_0=70$ \si{km.s^{-1}.Mpc^{-1}}.
Quoted results and figures repeated this procedure 10 times and averaged over each instance.

%\begin{figure*}[t]
%    \centering
%    \includegraphics[width=1.5\columnwidth]{figures/posterior.pdf}
%    \caption{Upper: Posteriors produced from individual events. The bias of posteriors with poorer localization is noticable, and is probably contributing to the underestimate of $H_0$ that we note. Lower: The posterior $p(H_0 | d_{GW}, d_C)$ produced after $100$ samples. The $68\%$ credible interval is shown by the dashed lines. The true injected value is shown in blue.}
%    \label{fig:posterior}
%\end{figure*}

\begin{figure}[t]
    \centering
    \includegraphics[width=\columnwidth]{figures/posterior.pdf}
    \caption{Upper: Posteriors produced from individual events. The bias of posteriors with poorer localization is noticable, and is probably contributing to the underestimate of $H_0$ that we note. Lower: The posterior $p(H_0 | d_{GW}, d_C)$ produced after $100$ samples. The $68\%$ credible interval is shown by the dashed lines. The true injected value is shown in blue. These results use cluster generation.}
    \label{fig:posterior}
\end{figure}

\begin{figure}
    \centering
    \includegraphics[width=\columnwidth]{figures/std.pdf}
    \caption{The standard deviation obtained from the posterior $p(H_0 | d_{GW}, d_C)$. The shaded region shows uncertainty from the population average. Estimates using catalogs with galaxy clustering are shown in red, while results from catalogs that do not use clustering are shown in blue.}
    \label{fig:std}
\end{figure}

\begin{figure}
    \centering
    \includegraphics[width=\columnwidth]{figures/diff.pdf}
    \caption{The deviation between the injected known value of $H_0$ and our expectation value based on $p(H_0 | d_{GW}, d_C)$. The shaded region shows uncertainty from the population average. Estimates using catalogs with galaxy clustering are shown in red, while results from catalogs that do not use clustering are shown in blue.}
    \label{fig:mean_diff}
\end{figure}

\begin{figure}[b]
    \centering
    \includegraphics[width=\columnwidth]{figures/correlation.pdf}
    \caption{The means and standard deviations of the estimates on $H_0$ for individual events. Note the clear correlation between these quantities. Estimates using catalogs with galaxy clustering are shown in red, while results from catalogs that do not use clustering are shown in blue.}
    \label{fig:correlation}
\end{figure}

The resulting posterior $p(H_0 | d_{GW}, d_C)$ produced after $100$ observations is shown in Figure \ref{fig:posterior} along with the posterior obtained from each observation considered in isolation.
In Figure \ref{fig:std} we show the standard deviation of the posterior $p(H_0 | d_{GW}, d_C)$ as a function of number of localized events. This did not significantly change when using the uniform and clustered catalogs.
%Some interesting correlations are noticable with events with wider spreads on the posterior tending to center around smaller values of $H_0$.
In Figure \ref{fig:mean_diff} we show the deviation of the posterior expectation on $H_0$ from the known value as a function of number of well localized events. We note that galaxy clustering introduces significant bias when compared with catalogs that assume a uniform distribution of galaxies.
We can see that our posterior produces an underestimate of $H_0$.
This suggests that there are as yet unidentified systematic effects that need to be taken into account.
Figure \ref{fig:correlation} shows a correlation between the means and standard deviations of the posterior distributions for single events.
This correlation would clearly bias our estimate of $H_0$ towards lower values, since these have smaller standard deviations and thus higher constraining power.
The origin of this correlation is therefore likely to be the unidentified systematic effect.
