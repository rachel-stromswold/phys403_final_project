\documentclass[a4paper, amsfonts, amssymb, amsmath, reprint, showkeys, nofootinbib, twoside]{revtex4-1}
\usepackage[english]{babel}
\usepackage[utf8]{inputenc}
\usepackage[colorinlistoftodos, color=green!40, prependcaption]{todonotes}
\input{preamble}
\usepackage[pdftex, pdftitle={Article}, pdfauthor={Author}]{hyperref} % For hyperlinks in the PDF
%\setlength{\marginparwidth}{2.5cm}
%\usepackage{biblatex}
%\addbibresource{Reference.bib}
\bibliographystyle{apsrev4-1}
\begin{document}
\title{Hubble constant "in tension"}

\author{noble-alligators}
    \email[Correspondence email address: ]{email@institution.com}% Your name
    \affiliation{Physics Department, University  of Rochester.}

\date{\today} % Leave empty to omit a date

\begin{abstract}
Abstract
\end{abstract}

\keywords{first keyword, second keyword, third keyword}


\maketitle

\section{\label{sec:intro} Introduction}

Hubble's law, the observation that galaxies are moving away from the Earth at a velocity proportional to their distance from the Earth, has been widely accepted since it was first proposed in the 1920s \cite{Hubble_1929}.
The constant of proportionality in this relationship is known as the Hubble parameter $H$.
The Hubble parameter in general changes with time, and therefore with redshift $z$. By convention, the Hubble parameter at the current time ($t=0$) is labeled $H_0$.
For objects relatively close to the Earth (i.e. those at low redshift), the redshift is linearly related to the distance $d$ by \cite{Freedman_2010}
\[
    c z = H_0 d.
\]

Variations in the physical assumptions underlying calculations of the Hubble constant, namely those used to compute the distances to objects at particular redshifts, have resulted in differing Hubble constant estimations.
Hubble’s initial estimation, determined by a simple linear fit between recession velocity and distance, gave roughly $500$ \si{km.s^{-1}.Mpc^{-1}} \cite{Hubble_1929}.
Subsequent measurements have progressively refined this estimate, eventually stabilizing at approximately $70$ \si{km.s^{-1}.Mpc^{-1}} \cite{Freedman_2010}.
However, different measurement approaches have led to significant disagreement between estimates: measurement uncertainties continue to diminish as methods improve, but the range of measured values does not.
Figure \ref{fig:hist_h0} gives an overview of recent measurements, showing significant disagreement.

\begin{figure}
    \centering
    \includegraphics[width=0.9\columnwidth]{H0whisker.pdf}
    \caption{A selection of recent $H_0$ measurements demonstrating the significant discrepancies in its value \cite{Pogosian_2020, Planck_2020, Aiola_2020, WMAP_2018, Henning_2018, Planck_2016, Hinshaw_2013, Freedman_2001, Freedman_2012, Riess_2016, Feeney_2018, Burns_2018,  Riess_2019, Camarena_2020, Riess_2021, Breuval_2021}.}
    \label{fig:hist_h0}
\end{figure}

There are two dominant methods for estimating the Hubble constant which are in tension with one another.
The first uses low redshift measurements of standard candles, e.g. type Ia Supernovae, in order to determine $H_0$.
The current estimate using this method is $H_0 = 74.03 \pm 1.42$ \si{km.s^{-1}.Mpc^{-1}} \cite{Riess_2019}.
Another approach, utilizing high redshift measurements of the cosmic microwave background (CMB), gives a value of $H_0 = 67.36 \pm 0.56$ \si{km.s^{-1}.Mpc^{-1}} \cite{Planck_2020}.

It is highly unlikely that the discrepancy between these measurements, at greater than $4\sigma$, can be explained as a statistical fluctuation or as the result of systematic uncertainties.
For instance, Tamara et al. state that even small systematic errors in redshift will have a significant impact on measurements of $H_0$; however, this is not enough to account for the tension in $H_0$ \cite{Davis_2019}.
See e.g. \cite{Efstathiou_2021,Calabrese_2008} for further discussion of systematic uncertainties.

A method for determining $H_0$ that does not rely on standard candles could potentially help resolve the tension in measurements of the Hubble constant.
In this paper we consider an approach utilizing gravitational waves as an independent distance measurement.

%\paragraph{Low redshift ($z \leq 10$) measurement}

%The $H_0$ is calculated locally by measuring the redshift of distant galaxies and then using a particular method to calculate their distances which is part of the cosmic distance ladder.
%The redshift can be easily measured, and the distances can be measured locally by getting the distance from the standard candle techniques, including the Type Ia Supernovae and Cepheid variables.
% \footnote{Astronomers use a “standard candle” to measure distances that are too vast to be measured using parallax. Because the light is spread out over a larger region, distant light sources appear fainter.}

%Tamara et. al gives a statement that even small systematic errors in redshift will result in a significant impact on $H_0$ measurements (2019)\cite{Davis_2019}; however, only considering those errors are not quite enough to resolve the $H_0$ tension people encountered. The research calculation results remain stable with constantly decreasing the error bar, as shown in the figure, except for the results obtained by Freedman et. al. in 2019 shows the relatively low $H_0$ value falls in $69.8 \pm 1.88$ \si{km.s^{-1}.Mpc^{-1}}\cite{Freedman_2019}. Instead of using Cepheid variables, they use the calibration of Tip of the Red Giant Branch(TRGB), which is parallel to but independent from the former one. TRGB samples have higher mass and less sensitivity to the metallicity, so that the potential systematic error in measurement has decreased.

%\paragraph{High redshift ($z \geq 10$) measurement}

%$H_0$ can be calculated using CMB temperature changes. Several characteristics, like the ratio of baryonic to dark matter and $H_0$, influence the specific shape of the curve (known as the acoustic power spectrum). The angular diameter distance to the last scattering surface is used to calculate $H_0$. That isn’t a direct observable; instead, trigonometry is used to infer it. The angular scale of the Baryon Acoustic Oscillations in the CMB may be directly measured, it’s the distance between troughs in the power spectrum is shown below.

%The temperature power spectrum is what is being used to determine the $H_0$. This is a way of looking at the ripples in the temperature field which encode the amount of dark and baryonic matter present, as well as the cosmological constant and other cosmological parameters.
%\footnote{All the parameters are simultaneously constrained with respect to each other using MCMC or other "fitting" approaches. }

%There are two possible systematic errors that are commonly seen. When we use two different likelihood pipelines for the data at certain multipoles, with different parameters used for the calibration efficiencies, it has little effect in reducing the Hubble tension \cite{Efstathiou_2021}. Therefore, the choice of likelihood will result in systematic errors. 
%The second is the systematic error that can occur in the lensing parameter\cite{Calabrese_2008}. The lensing parameter simply rescales by hand the effects of gravitational lensing on the CMB angular power spectra and can be measured by the smoothing of the peaks in the damping tail. This lensing anomaly is not seen in the
%Planck trispectrum data \footnote{CMB lensing} that offer a complementary and independent measurement. If there is no new physics in it, the alternative explanation could be due to a small but still undetected systematic error in the Planck data which can be used to reduce the Hubble Tension.
  %I believe leaving the sections in separate files is more organized, change it if you desire 
\section{\label{sec:methods} Methods}


Binary black hole mergers are often called ``dark sirens'' due to their lack observable EM counterparts.
Though the detection of an EM counterpart can facilitate the determination of host galaxy and therefore result in a well-constrained measurement of $H_0$ \cite{GW170817_H0}, such detections are rare: at time of writing, GW170817 is the only neutron start merger detection with an identified EM counterpart \cite{GW170817_announce}.

As early as 1986 it was proposed that an array of multiple gravitational wave detectors could be used to estimate the distance to an event $d_L$ and bounds on sky location \cite{Schutz_1986}.
Gravitational waves originating from a binary black hole system during the inspiral phase of the merger can be used as dark sirens due to the fact that the individual black hole masses can be determined from the gravitational wave frequency.
The power radiating from the binary system is due to its orbital energy, and therefore the power radiating from the source can be determined purely from the gravitational wave frequency, without any knowledge of the luminosity distance.
The power detected is related to this emitted power through an inverse square law, and so the luminosity distance can be determined without the need for a cosmic distance ladder.
Gravitational waves therefore present a completely independent method by which we can determine distances to galaxies, provided we can determine the host galaxy for a given merger \cite{GW170814_DES,GW170817_H0,Nair_2018}.

Since we can estimate the sky location from which the signal was emitted along with the distance to the event, these estimates can then be used to produce a catalog of candidate host galaxies from existing sky surveys.
Based on the posterior distributions for these localizations, we use a method for measuring $H_0$ layed out by Nair et al. \cite{Nair_2018}.
It has been predicted that these methods can confine the Hubble constant within the decade \cite{Chen_2018}.
This technique was applied to the event GW170814 by collaborators from the LIGO and Virgo teams using results from the Dark Energy Survey of the southern sky \cite{GW170814_DES}.
We carry out a similar analysis on simulated data sets based on historic GW data and simple physical assumptions for the distribution of galaxies within clusters.

%\section{Data} \label{sec:develop}

Historic data is taken from merger catalogs from the second and third year of LIGO observations \cite{GWTC_2,GWTC_3}. Our analysis is only interested in the solid angle covered by sky localization ($\Omega$), distances obtained from the GW data ($d_L$), and their uncertainties, $\sigma_d$. We thus marginalize over all other parameters to obtain a three dimensional probability distribution containing only these parameters of interest. It vastly simplifies our analysis if we assume that the uncertainties on $d_L$ are Gaussian. However, posterior distributions from the LIGO papers contain differing upper and lower credible limits. We take the half width on these posteriors as our Gaussian uncertainties.

After obtaining these marginalized probability distributions, we use historic data as a training set for Gaussian kernel density estimation. We then use the kernel estimation to sample simulated events which have the same underlying probability distribution as observations. The returned samples have a finite probability to be from unphysical regions, i.e. $d_L \leq 0$, $\Omega\leq 0$ or $\sigma_d > d_L$. To counteract this, we reject all samples which are outside of the extremal historic values or violate the $\sigma_d \leq d_L$ constraint. Many of the sampled events will have very large localization regions, which makes it inconvenient to use them for estimates of $H_0$. Since our goal is to produce an estimate of how many observations are required to produce estimates of $H_0$ we only perform further analysis on events that were well localized.

In principle, one should consider the posterior distribution on sky location which has a finite probability to be found outside the credible interval. To simplify our analysis, we approximate this probability distribution as uniform within the localization volume and zero everywhere else. This treatment has precedent in existing studies \cite{Nair_2018, GW170814_DES}.

After producing simulated GW events and their corresponding localization volumes, we need to select a suitable sky catalog. These data are also simulated by randomly sampling a number of galaxy clusters which are assumed not to interact with one another. The total mass of the cluster is assumed to follow a Press-Schechter mass distribution\cite{Press_1974}. Peculiar motions (which affect the measured redshift $z$) and distance from the center of mass are sampled such that the Virial theorem is obeyed, but other important factors like the very high concentrations of dark matter within clusters are neglected. More details on cluster generation are given in appendix \ref{sec:clust_gen}.
    
%\subsection{\label{Method} Methodology}

We start with a simple application of Bayes' theorem,
\begin{align}
    p(H_0|d_{GW}, d_C)&\propto p(d_{GW}, d_C|H_0)p(H_0)\nonumber\\
    &= p(d_{GW}|H_0)p(d_C|H_0)p(H_0).
\label{eq:bayes}
\end{align}
where $d_{GW}$ refers to the observed data from the gravitational wave detectors and $d_C$ refers to data from the known catalog of stars and their redshifts. We break this joint likelihood into a product of likelihoods since both experiments are statistically independent.

Since we are considering dark sirens without electromagnetic counterparts, we do not know the host galaxy in which the merger occurred. Thus we must marginalize over all possible galaxies within the volume of possible locations supplied by the GW data. Since the number of galaxies is discrete, this is expressed as a sum:

\begin{align}
    p(&d_{GW}, d_C|{z_j, \hat{\Omega}_j},H_0)  \propto  \nonumber\\
    &\sum_{i} w_i \int dV p(d_{GW}|d_L, \hat{\Omega}_{GW}) 
    p(d_C|{z_, \hat{\Omega}_j}) \nonumber\\
    & \delta_D (d_L - d_L(z_i, H_0)) \delta_D (\hat{\Omega}_{GW} - \hat{\Omega}_i).
    \label{eq:marginal_like}
\end{align}
In principle, we should apply a weight to each galaxy $w_i$ we include this in the expression below, but for simplicity we take $w_i=1$ for all galaxies within the volume of interest and $w_i=0$ for all galaxies outside. This same approximation has been used in previous analyses \cite{Chen_2018,GW170814_DES,Nair_2018}. To evaluate this integral we switch to spherical coordinates and work in redshift space as opposed to distance space. This results in a change of coordinates

\begin{equation}
dV \propto \frac{d^2 V}{d z_i d\hat{\Omega}_i} \frac{d z_i}{d r} d r d\hat{\Omega} \propto \frac{r^2(z_i)}{H(z_i)}
\end{equation}
where $H(z_i)$ is the Hubble parameter in redshift space,

\begin{equation}
H(z_i) = H_0^3 \left(\Omega_m (1+z)^3 + \Omega_\Lambda\right)
\end{equation}
where we have assumed a flat universe so $\Omega_k = 0$. We will take $\Omega_m = 0.3$ and $\Omega_\Lambda = 0.7$ in keeping with Soares-Santos et al \cite{GW170814_DES}. We expect our catalog of galaxies to have very low uncertainty on sky location, so we take $p(d_C|z_i,\Omega)=p(d_C|z_i)\delta(\Omega - \Omega_i)$. Finally we can write the posterior distribution for $H_0$:

\begin{align}
     p&(H_0|d_{GW}, d_C)\propto p(H_0) \sum_i \frac{1}{Z_i} \nonumber\\
     &\times \int dz_i p\left(d_{GW}|d_L(z_i, H_0), \hat{\Omega_{i}}\right)p(d_C|z_i)\frac{r^2 (z_i)}{H(z_i)}
    \label{eq: finaleq}
\end{align}
where $d_L(z_i, H_0) = cz_i / H_0$ uses Hubble's law to express luminosity distance in terms of redshift and an assumed value for $H_0$. We assume that the errors on redshifts are Gaussian, in keeping with the analysis by Singer et al. \cite{Singer_2016}. Under this assumption,

\begin{align}
    p&\left(d_{GW}|d_L(z_i, H_0)\right)\propto \nonumber\\
    &\frac{p(\hat{\Omega})}{\sqrt{2\pi}\sigma(\hat{\Omega})} \exp\left(-\frac{(cz_i/H_0 - \hat{d}_{GW})^2}{2\sigma_{GW}^2}\right).
    \label{eq:GW_like}
\end{align}

\section{Results} \label{sec:conclusions}
We produce simulated events until we find at least $100$ with sufficiently small localizations that further analysis is practical. This cutoff is arbitrary, for all results reported we used a maximum volume of $0.5\times 10^{9}$\si{Mpc^3}. We also observed values which had anomalously small localization volumes so we also constrained all simulated events to have a localization volume larger than $0.5\times 10^{7}$\si{Mpc^3}. We found that a total of $650$ observations were needed to produce the desired $100$ well localized events. These localized regions corresponded to $15.43\%$ of all generated samples. In all of our tests we injected a known value of $H_0=70$\si{km.s^{-1}.Mpc^{-1}}.

The resulting posterior $p(H_0 | d_{GW}, d_C)$ produced after $100$ observations is shown in Figure \ref{fig:posterior} along with the posterior obtained from each observation considered in isolation. In Figure \ref{fig:std} we show the standard deviation of the posterior $p(H_0 | d_{GW}, d_C)$ as a function of number of localized events. Some interesting correlations are noticable with events with wider spreads on the posterior tending to center around smaller values of $H_0$. In Figure \ref{fig:mean_diff} we show the deviation of the posterior expectation on $H_0$ from the known value as a function of number of well localized events. We can see that our posterior produces an underestimate of $H_0$. This suggests that there are as yet unidentified systematic effects that need to be taken into account. 

\begin{figure}
    \centering
    \includegraphics[width=0.95\columnwidth]{figures/diff.pdf}
    \caption{The deviation between the injected known value of $H_0$ and our expectation value based on $p(H_0 | d_{GW}, d_C)$.}
    \label{fig:mean_diff}
\end{figure}

\begin{figure}
    \centering
    \includegraphics[width=0.95\columnwidth]{figures/std.pdf}
    \caption{The standard deviation obtained from the posterior $p(H_0 | d_{GW}, d_C)$.}
    \label{fig:std}
\end{figure}

\begin{figure}
    \centering
    \includegraphics[width=0.95\columnwidth]{figures/posterior.pdf}
    \caption{Upper: Posteriors produced from individual events. The bias of posteriors with poorer localization is noticable, and is probably contributing to the underestimate of $H_0$ that we note. Lower: The posterior $p(H_0 | d_{GW}, d_C)$ produced after $100$ samples. The $68\%$ Confidence interval is shown by the dashed lines. The true injected value is shown in blue.}
    \label{fig:posterior}
\end{figure}

\input{sections/conclusion.tex}
\input{sections/acknowledgements.tex}

\begin{thebibliography}{4}
\bibitem{Griffiths}
D. J. Griffiths,
\textit{Introduction to Electrodynamics}
(Cambridge University Press, Cambridge, 2017).

\bibitem{Plank2018}
Aghanim, N., Akrami, Y., Ashdown, M., Aumont, J., Baccigalupi, C., Ballardini, M., Banday, A. J., Barreiro, R. B., Bartolo, N., Basak, S., Battye, R., Benabed, K., Bernard, J.-P., Bersanelli, M., Bielewicz, P., Bock, J. J., Bond, J. R., Borrill, J., Bouchet, F. R., … Zonca, A. (2020). Planck 2018 results. VI. Cosmological parameters. Astronomy &amp; Astrophysics, 641. https://doi.org/10.1051/0004-6361/201833910 .

\bibitem{Feynman}
R. P. Feynman, R. B. Leighton and M. Sands,
\textit{Lições de Física de Feynman}
(Editora Bookman, Porto Alegre, 2008).

\bibitem{Jackson-CE}
J. D. Jackson,
\textit{Classical Electrodynamics}
(John Wiley \& Sons, Danvers, 1999).
\end{thebibliography}

\appendix*
\input{sections/appendix1.tex}

\end{document}
